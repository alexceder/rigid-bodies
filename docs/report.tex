%\documentclass[a4paper,12pt,twoside]{report}
\documentclass[a4paper,12pt]{report}

\usepackage{url}
\usepackage{amsmath}
\usepackage{lipsum}
\usepackage{listings}
\usepackage{color}
\usepackage{graphicx}
\usepackage[font=footnotesize, margin=15pt]{caption}

\definecolor{mygreen}{rgb}{0,0.6,0}
\definecolor{mygray}{rgb}{0.6,0.6,0.6}
\definecolor{mymauve}{rgb}{0.58,0,0.82}

\lstset{
  captionpos=b,
  numbers=left,
  frame=tb,
  showstringspaces=false,
  numbersep=15pt,
  xleftmargin=30pt,
  aboveskip=15pt,
  belowskip=15pt,
  keywordstyle=\color{blue},
  stringstyle=\color{mymauve},
  numberstyle=\color{mygray},
  rulecolor=\color{mygray}
}

\title{Resolution of two dimensional collisions}
\author{Philip Burridge\\
        Dan Camarda\\
        Alexander Cederblad\\
        Rasmus Haapaoja\\
        Fredrik Johnson}
\date{\today}


%-- Document
\begin{document}


%-- Title
\maketitle


%-- Abstract
\begin{abstract}
This report will cover the way of thinking when creating a physics simulation for collision impulses for basic geometries.
\end{abstract}


%-- Table of Contents
\tableofcontents
\addtocontents{toc}{\protect\thispagestyle{empty}}
%\listoffigures
%\listoftables
%\lstlistoflistings


%-- Introduction
\chapter{Introduction}
\setcounter{page}{1}

We have chosen a general system involving collision impulse between rigid bodies in two and three dimensions. The implementation of the problem of resolving collisions includes calculating collisions between these bodies. Because of this and the fact that the focus of this project is the physical system and the numerical methods of simulation the simplest geometries was chosen to narrow the focus. What is the simplest geometries for collision detection, you say? And circles and spheres are the answer to your question. Since they can only collide at \emph{one} point with other convex geometries they are the best choice to keep the focus on the simulation. Figure ~\ref{fig:snapshot} show a snapshot of this kind of simulation with circles.

\begin{figure}[!h]
    \centering
    \includegraphics[width=0.8\textwidth]{figures/snapshot.png}
    \caption{A snapshop of a simulation of circles interacting with each other. The line from the origin indicates the angle.}
    \label{fig:snapshot}
\end{figure}


%-- Method
\chapter{Method}

% Collision detection
\section{Collision detection}

To be able to resolve a collision an actual collision must be registered. As mentioned in previous chapter the focus is not put on collision detection. There are some information that we need to know about a collision to be able to resolve the collision impulse in a later step. We need to know the \emph{penetration} of the collision, the \emph{normal} of the collision and the \emph{collision point}. Figure ~\ref{fig:snapshot} displays a collision between two circles including this information.

\begin{figure}[!h]
    \centering
    \includegraphics[width=0.8\textwidth]{figures/snapshot.png}
    \caption{A collision including the normal of the collision and other quantities related to the collision.}
    \label{fig:collision}
\end{figure}

% Resolvning an impulse
\section{Resolving an impulse}

To explain how to resolve an impulse\cite{gdm} let us first think about to object colliding with each other both with their own velocity. Imagine that we know the normal of the collision. During a collision an instant force, \emph{impulse}, will act on both object knocking them apart from each other. This change in the objects relative velocity as explained in \ref{eq:1}, tells us that the relative velocity will be inverted proportional with the \emph{coefficient of restitution}, \emph{e},  (which will be covered later on).

\begin{equation}
\mathbf v'_{AB}\cdot \mathbf n=-e\mathbf v_{AB}\cdot \mathbf n
\label{eq:1}
\end{equation}

The above equation is used, among some other physical properties, to solve a coefficient \ref{eq:2} for the impulse that later will be applied to the objects as a way of changing their velocity and angular velocity.

\begin{equation}
j = \dfrac{ -(1+e) \mathbf v_{AB} \cdot \mathbf n }{
    \mathbf n \cdot \mathbf n ( \dfrac{1}{M_{A}} + \dfrac{1}{M_{A}} )
    + \dfrac{ (\mathbf r_{AP\perp} \cdot \mathbf n)^2}{I_{A} }
    + \dfrac{ (\mathbf r_{BP\perp} \cdot \mathbf n)^2}{I_{B} } }
\label{eq:2}
\end{equation}

Now that the impulse coefficient is known, one might think how is a scalar going to help us resolve a collision impulse. The question is legitimate and it will be explained in the following equations \ref{eq:3} and \ref{eq:4}. The scalar is multiplied with the normalized \emph{collision vector} in both equations.

\begin{equation}
\mathbf v'_{A}=\mathbf v_{A}+\dfrac{j}{M_{A}}\mathbf n
\label{eq:3}
\end{equation}
\begin{equation}
\omega'_{A}=\omega_{A}+\dfrac{\mathbf r_{AP\perp}\cdot j\mathbf n}{I_{A}}
\label{eq:4}
\end{equation}

This is basically it; when a collision occurs, figure out some properties of the collision then apply the impulse to the objects velocity and angular velocity.

% Resolvning an impulse
\section{Numerical integration}

Since computers live in a world of zeroes and ones we need some way to integrate numerically, the most basic and widely used one is the \emph{Euler method}\cite{gdm}. For this kind of simulation Euler is enough for most applications, however another method will also be covered.
The \emph{Euler method} \ref{eq:5} works by adding the current state with its derivative multiplied by the step in time.

\begin{equation}
y_{n+1}=y_{n}+hf(t_n, y_n)
\label{eq:5}
\end{equation}

For example \ref{eq:6} if we are going to integrate a velocty for the next time step, grab the current velocity and multiply it by the acceleration multiplied with the time step.

\begin{equation}
v(t+1)=v(t)+ha(t)
\label{eq:6}
\end{equation}

Since this method is as basic as it gets, keep in mind that an error is going to be added each iteration.

% OpenGL Implementation
\section{OpenGL implementation}

The chosen method to run the simulation is \emph{OpenGL} which is a computer graphics API for most graphics cards. Our implementation was written in \emph{C++}. Listing \ref{lst:1} shows pseudo code for the implementation.

\begin{lstlisting}[caption={Pseudo code for the simulation loop.}, label=lst:1]
void mainLoop()
{
    while (time < INFINITY)
    {
        calculatePhysics();
        drawScene();
    }
}
\end{lstlisting}

\subsection{Running the simulation}
Navigate to the folder with your current operating system and click the executable file named $simulation$ to run the simulation. The software is not tested on any Linux distribution.

\subsubsection{Requirements}
\begin{itemize}
    \item Graphics card from this century
\end{itemize}


%-- Discussion
\chapter{Discussion}

We though that it was a bit to easy. Maybe we should have challenged ourselves a bit more.


%-- Bibliography
\bibliographystyle{vancouver}
\bibliography{report}


%-- End of Document
\end{document}

